\makerubrichead{Projects}

\begin{itemize}
  \item {
    \textbf{ACS Future School} \\
    Source: Closed \\
    Education platform built on NextJS using microservice architecture. It uses ExpressJS and Python microservices.
  }
  \item {
      \textbf{Authentidocs} \\
      Source: \href{https://github.com/AuthentiDocs/authentidocs}{github.com/AuthentiDocs/authentidocs} \\
      A simple website for doing collaborative work where each file and interaction is signed digitally built on Sveltekit and PostgreSQL.
    }
  \item {
    \textbf{EPathshala Frontend} \\
    Source: \href{https://github.com/Siam11651/ePathshala}{github.com/Siam11651/ePathshala} \\
    Frontend of EPathshala. It is an academic database project. Frontend built raw HTML, CSS and Javascript.
  }
  \item {
    \textbf{EPathshala Backend} \\
    Source: \href{https://github.com/Siam11651/ePathshala-Backend}{github.com/Siam11651/ePathshala-Backend} \\
    Backend of EPathshala. Backend built on Drogon, a web framework built on C++. Used PostgreSQL database.
  }
  \item {
      \textbf{Cat vs Dog ML Trainer} \\
      Source: \href{https://github.com/Siam11651/cat_vs_dog_ml_trainer}{github.com/Siam11651/cat\_vs\_dog\_ml\_trainer} \\
      Used Tensorflow to train \href{https://www.kaggle.com/competitions/dogs-vs-cats}{this} dataset
    }
  \item {
    \textbf{Cat vs Dog Classifier} \\
    Source: \href{https://github.com/Siam11651/cat_vs_dog_classifier}{github.com/Siam11651/cat\_vs\_dog\_classifier} \\
    Android app made in Java to infer on real time whether the image contains a cat or a dog.
  }
  \item {
    \textbf{Ludo} \\
    Source: \href{https://github.com/Siam11651/ludo}{github.com/Siam11651/ludo} \\
    A simple cross-platform implementation to ludo for desktop using OpenGL 4.6.
  }
  \item {
    \textbf{Ludo Android} \\
    Source: \href{https://github.com/Siam11651/ludo-android}{github.com/Siam11651/ludo-android} \\
    Port to ludo for android. It uses native android activity implemented in C++. Uses OpenGL ES 3.2. Just had to modify the shaders.
  }
  \item {
    \textbf{HTPP (Incomplete)} \\
    Source: \href{https://github.com/Siam11651/htpp}{github.com/Siam11651/htpp} \\
    A simple http/1.1 server implemented in C++. Currently it uses seperate threads to handle seperate requests. I do intend to use coroutines to reduce thread overheads. Only cookie parser and OPTIONS request handler left to implement to make it v1.0.
  }
\end{itemize}