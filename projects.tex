\makerubrichead{Projects}

\begin{itemize}
  \item {
    \textbf{ACS Future School} (\href{https://www.acsfutureschool.com}{acsfutureschool.com}) \\
    Source: Closed \\
    Education platform built on \textbf{NextJS} using \textbf{microservice} architecture. It uses \textbf{ExpressJS} and \textbf{Python} microservices. \textbf{Developing the frontend} using \textbf{TailwindCSS} and \textbf{NextJS}. \textbf{Developed dashboards} for clients, customers, and admins with integrated features like quizzes and videos. And also handling quizes on client side.
  }
  \item {
    \textbf{Beauty Trends} (\href{https://www.beautytrendsbd.com}{beautytrendsbd.com}) \\
    Source: \href{https://github.com/1905039/beauty-trends}{github.com/1905039/beauty-trends} \\
    E-Commerce platform built on \textbf{Sveltekit} and \textbf{PostgreSQL} database. \textbf{Developed both the frontend and backend} of the project. Implemented UI using \textbf{Bootstrap} theme, and \textbf{SCSS} to layout pages. Implemented backend using \textbf{Sveltekit} API handlers. Write seperate \textbf{PostgreSQL} function for each API to handle concurrency and minimize bandwidth between \textbf{Sveltekit} server and database server.
  }
  \item {
      \textbf{Authentidocs} \\
      Source: \href{https://github.com/AuthentiDocs/authentidocs}{github.com/AuthentiDocs/authentidocs} \\
      This is an academic project. A simple website for doing collaborative work where each file and interaction is signed digitally built on \textbf{Sveltekit} and \textbf{PostgreSQL}. \textbf{Developed the frontend} using \textbf{Flowbite} and \textbf{TailwindCSS}.
    }
  \item {
    \textbf{EPathshala Frontend} \\
    Source: \href{https://github.com/Siam11651/ePathshala}{github.com/Siam11651/ePathshala} \\
    Frontend of EPathshala. It is an academic database project. Frontend built \textbf{raw HTML}, \textbf{CSS} and \textbf{Javascript}.
  }
  \item {
    \textbf{EPathshala Backend} \\
    Source: \href{https://github.com/Siam11651/ePathshala-Backend}{github.com/Siam11651/ePathshala-Backend} \\
    Backend of EPathshala. Backend built on \textbf{Drogon}, a web framework built on \textbf{C++}. Used \textbf{PostgreSQL} database.
  }
  \item {
      \textbf{Cat vs Dog ML Trainer} \\
      Source: \href{https://github.com/Siam11651/cat_vs_dog_ml_trainer}{github.com/Siam11651/cat\_vs\_dog\_ml\_trainer} \\
      Used \textbf{Tensorflow} to train \href{https://www.kaggle.com/competitions/dogs-vs-cats}{this} dataset. It uses a very simple convolutional neural network.
  }
  \item {
    \textbf{Cat vs Dog Classifier} \\
    Source: \href{https://github.com/Siam11651/cat_vs_dog_classifier}{github.com/Siam11651/cat\_vs\_dog\_classifier} \\
    \textbf{Android} app built on \textbf{Java} to infer on real time whether the image contains a cat or a dog. It uses seperate threads to show camera preview and infer the image, as image inference takes 300ms. It helps prevent slowdown of preview.
  }
  \item {
    \textbf{Ludo} \\
    Source: \href{https://github.com/Siam11651/ludo}{github.com/Siam11651/ludo} \\
    A simple cross-platform implementation to ludo for desktop using \textbf{OpenGL 4.6}. Used \textbf{GLSL shader} programs to process textures and transform objects.
  }
  \item {
    \textbf{Ludo Android} \\
    Source: \href{https://github.com/Siam11651/ludo-android}{github.com/Siam11651/ludo-android} \\
    Port to ludo for \textbf{android}. It uses \textbf{native android activity} implemented in \textbf{C++}. Uses \textbf{OpenGL ES 3.2}. Just had to modify the \textbf{GLSL shaders} to be compatible with \textbf{OpenGL ES 3.2}.
  }
  \item {
    \textbf{Safe n Secure Wheelchair Buddy} \\
    Source: \href{https://github.com/Siam11651/safe-n-secure-wheelchair-buddy}{github.com/Siam11651/safe-n-secure-wheelchair-buddy} \\
    We made a wheelchair to help prevent accidents. That can be connected with a phone. This app built on \textbf{Java} helps maneuver the wheelchair. It uses bluetooth to communicate.
  }
  \item {
    \textbf{HTPP (Incomplete)} \\
    Source: \href{https://github.com/Siam11651/htpp}{github.com/Siam11651/htpp} \\
    A simple \textbf{http/1.1} server implemented in \textbf{C++}. Currently it uses seperate threads to handle seperate requests. I do intend to use coroutines to reduce thread overheads. Only cookie parser and OPTIONS request handler left to implement to make it v1.0.
  }
  \item {
    \textbf{Pumpa} \\
    Source: \href{https://github.com/Siam11651/Pumpa}{github.com/Siam11651/Pumpa} \\
    A simple note keeping app build with \textbf{JavaFX}.
  }
\end{itemize}